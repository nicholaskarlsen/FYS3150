\documentclass[10pt,showpacs,preprintnumbers,footinbib,amsmath,amssymb,aps,prl,twocolumn,groupedaddress,superscriptaddress,showkeys]{revtex4-1}
\usepackage{graphicx}
\usepackage{dcolumn}
\usepackage{bm}
\usepackage[colorlinks=true,urlcolor=blue,citecolor=blue]{hyperref}
\usepackage{color}
\usepackage{listings}
\usepackage{subfig}

\lstset{ %
  basicstyle=\footnotesize,        % the size of the fonts that are used for the code
  breakatwhitespace=false,         % sets if automatic breaks should only happen at whitespace
  breaklines=true,                 % sets automatic line breaking
  captionpos=t,                    % sets the caption-position to bottom
  deletekeywords={...},            % if you want to delete keywords from the given language
  escapeinside={\%*}{*)},          % if you want to add LaTeX within your code
  extendedchars=true,              % lets you use non-ASCII characters; for 8-bits encodings only, does not work with UTF-8
  frame=single,                    % adds a frame around the code
  keepspaces=true,                 % keeps spaces in text, useful for keeping indentation of code (possibly needs columns=flexible)
 % language=Python,                 % the language of the code
  morekeywords={*,...},           % if you want to add more keywords to the set
  numbers=left,                    % where to put the line-numbers; possible values are (none, left, right)
  numbersep=5pt,                   % how far the line-numbers are from the code
  showspaces=false,                % show spaces everywhere adding particular underscores; it overrides 'showstringspaces'
  showstringspaces=false,          % underline spaces within strings only
  showtabs=false,                  % show tabs within strings adding particular underscores
  stepnumber=1,                    % the step between two line-numbers. If it's 1, each line will be numbered
  tabsize=2,                       % sets default tabsize to 2 spaces
}


\begin{document}
\title{FYS3150 Computational Physics - Project 3}
\author{Nicholas Karlsen}

\begin{abstract}
This is an abstract
\end{abstract}

\maketitle

\section{Introduction}
  
  Lastly, the source code for any code discussed in this report can be found on my
  Github at: \url{https://github.com/nicholaskarlsen/FYS3150}

\section{Theory, Algorithms and Methods}
  
  \subsection{Newton's law of universal gravitation}
    Between every body, there is a force of attraction inversly proportional to the square of the separation, or more precisely, the force acting on some body with mass $m$ due to a mass $m'$ is
    \begin{equation}
      \mathbf F = -G\frac{m m'}{|\mathbf r - \mathbf r'|^2}\mathbf{\hat{u}_{r-r'}}, \quad \mathbf{\hat{u}_{r-r'}} = \frac{\mathbf r - \mathbf r'}{\mathbf |\mathbf r - \mathbf r'|}
    \end{equation}
    Where $G$ is the gravitational constant and $\mathbf r, \mathbf r'$ denote the position vectors of bodies with mass $m, m'$ respectively.

    Chosing the the 2D cartesian coordinate system, let $\mathbf r - \mathbf r'= (x_{r}, y_{r})$, where the $r$ suffix denote that these coordinates are to be understood as the relative coordinates between bodies $m, m'$.
    Further, $|\mathbf r - \mathbf r'| = \sqrt{x_r^2 + y_r^2} = r$ is the distance between the two bodies.

    By Newtons second law, the acceleration on body 1 due to the gravitational pull of body 2 can then be written as
    \begin{equation}
      \mathbf a = \frac{1}{m}\mathbf F = -G \frac{m'}{r^2}\frac{(\mathbf r-\mathbf r')}{r} = -G\frac{m'}{r^2}\frac{\left(x_r, y_r\right)}{r}
    \end{equation}
    Written out component-wise in terms of the positions, we get the set of coupled differential equations

    \begin{equation}
      \frac{\partial^2 x}{\partial t^2} = -G\frac{m'}{r^2}\frac{x_r}{r}, \quad
      \frac{\partial^2 y}{\partial t^2} = -G\frac{m'}{r^2}\frac{y_r}{r}
    \end{equation}
    Similar, for 3 dimensions where $\mathbf r - \mathbf r' = (x_r, y_r, z_r), r=\sqrt{x_r^2 + y_r^2 + z_r^2}$.

    \subsubsection{Relativistic Correction}

      \begin{figure}[h!]
        \center
        \includegraphics[width=4cm]{figs/486px-Perihelion-Aphelion.png}
        \caption{In an eliptic orbit, the closest and farthest points in the orbit is defined as the Perhelion and Aphelion respectively [\href{https://en.wikipedia.org/wiki/Perihelion_and_aphelion}{Image source}]}
        \label{fig:perhelion}
      \end{figure}

      The aforementioned model of gravity fails to predict the perihelion (see fig. \ref{fig:perhelion}) precession of mercury, which is observed to be $43"$ per century \cite{problem_set}. 

      That is, the closed, uniform elliptical orbits predicted by the Newtonian model for gravity does not match with observations in Astronomy, where the perihelion of Mercury seems to shift over time. In fact, the perihelion precession of mercury was the first experimental confirmations of General relativity, which accurately predicts this phenomena.

      And so, a correcting factor accounting for the relativistic effects is added to Newtons model, and the magnitude of the gravitational force becomes \cite{problem_set}

      \begin{equation}
        |\mathbf F| = G \frac{m m'}{r^2}\left[1 + \frac{3l^2}{r^2c^2}\right]
      \end{equation}
      Where $l$ denotes the magnitude of the angular momentum of the orbiting body and $c$ the speed of light.

      Now, in order to find the perihelion precession we define a coordinate system such that 

\subsection{Solving ODE numerically}
  \subsubsection{Forward Euler}
    Consider a function $y(t)$, which derivative, $y'(t, y)$ has a known form.

  \begin{lstlisting}[mathescape=true, language=python, title=Velocity-Verlet Algorithm]
  for $i = 0, \dots, N-1$
    $\mathbf r_{i+1} = \mathbf r_i + \mathbf v_i \Delta t + \frac{1}{2}\mathbf a_i(\Delta t)^2$
    $\mathbf v_{i+1} = \mathbf v_i + \frac{1}{2}(\mathbf a_{i+1} + \mathbf a_i)\Delta t  $
  \end{lstlisting}

    \begin{lstlisting}[mathescape=true, language=python, title=Forward Euler Algorithm]
  for $i = 0, \dots, N-1$
      $\mathbf v_{i + 1} = \mathbf v_{i} + \mathbf a_{i}\Delta t$
      $\mathbf r_{i + 1} = \mathbf r_{i} + \mathbf v_{i}\Delta t$
  \end{lstlisting}

  \begin{lstlisting}[mathescape=true, language=python, title=Euler-Cromer Algorithm]
  for $i = 0, \dots, N-1$
      $\mathbf v_{i + 1} = \mathbf v_{i} + \mathbf a_{i}\Delta t$
      $\mathbf r_{i + 1} = \mathbf r_{i} + \mathbf v_{i + 1}\Delta t$
  \end{lstlisting}

\section{Results and Discussions}
  \subsection{Object orientation}
    When designing the class \lstinline{solarsystem}, i wanted to strike a balance between utilizing the benefits, and expandability you get from object orientation whilst also not abstracting the data too much. As such, the class simply manipulates arrays of a particular format in a particular way. Not abstracting the process by creating objects for each planet (or something else of that nature), which seems to be a pitfall of object orientation as i percieve it.

    A particular benefit in the way i wrote my code, is that there is no difference if the input data is in 2 Dimensions or 3. The code will work just the same either way by making full use of the way Numpy arrays work.


\onecolumngrid
\begin{figure}[h!p]
  \center
  \subfloat[][Velocity-Verlet]{\includegraphics[width=4cm]{figs/ex_b_orbit_velocityverlet_1.pdf}}
  \subfloat[][Forward Euler]{\includegraphics[width=4cm]{figs/ex_b_orbit_eulerforward_1.pdf}} 
  \subfloat[][Velocity-Verlet]{\includegraphics[width=4cm]{figs/ex_b_orbit_velocityverlet_2.pdf}}
  \subfloat[][Forward Euler]{\includegraphics[width=4cm]{figs/ex_b_orbit_eulerforward_2.pdf}} 
  \\
  \subfloat[][Velocity-Verlet]{\includegraphics[width=4cm]{figs/ex_b_orbit_velocityverlet_3.pdf}}
  \subfloat[][Forward Euler]{\includegraphics[width=4cm]{figs/ex_b_orbit_eulerforward_3.pdf}} 
  \subfloat[][Velocity-Verlet]{\includegraphics[width=4cm]{figs/ex_b_orbit_velocityverlet_4.pdf}}
  \subfloat[][Forward Euler]{\includegraphics[width=4cm]{figs/ex_b_orbit_eulerforward_4.pdf}} 
  \\
  \subfloat[][Velocity-Verlet]{\includegraphics[width=4cm]{figs/ex_b_orbit_velocityverlet_5.pdf}}
  \subfloat[][Forward Euler]{\includegraphics[width=4cm]{figs/ex_b_orbit_eulerforward_5.pdf}}
  \subfloat[][Velocity-Verlet]{\includegraphics[width=4cm]{figs/ex_b_orbit_velocityverlet_6.pdf}}
  \subfloat[][Forward Euler]{\includegraphics[width=4cm]{figs/ex_b_orbit_eulerforward_6.pdf}} 
  \caption{The orbit of earth around a stationary sun for a timeperiod of 10 Years with simulated with a varying number of integration points N (see fig titles) using the Velocity-Verlet and Forward Euler algorithms}
\end{figure}
\twocolumngrid
 
\begin{figure}[h!]
  \center
  \subfloat[][Velocity-Verlet]{\includegraphics[width=8cm]{figs/ex_b_speed_velocityverlet_6.pdf}\label{fig:earthsun energy verlet}}\\
   \subfloat[][Forward Euler]{\includegraphics[width=8cm]{figs/ex_b_speed_eulerforward_6.pdf}\label{fig:earthsun energy forward euler}}
   \caption{The fluctuation of the total energy of the Earth in the Earth-Sun system for solutions using the Velocity-Verlet algorithm (a) and the Forward Euler algorithm (b) in a 10 Year simulation}
   \label{fig:c_energy}
\end{figure}

\begin{figure}[h!]
  \center
  \subfloat[][Velocity-Verlet]{\includegraphics[width=8cm]{figs/ex_b_radius_velocityverlet_6.pdf}\label{fig:earthsun potential verlet}}\\
   \subfloat[][Forward Euler]{\includegraphics[width=8cm]{figs/ex_b_radius_eulerforward_6.pdf}\label{fig:earthsun potential forwardeuler}}
   \caption{The fluctuation of the total energy of the Earth in the Earth-Sun system for solutions using the Velocity-Verlet algorithm (a) and the Forward Euler algorithm (b) in a 10 Year simulation}
   \label{fig:c_energy}
\end{figure}

\begin{figure}[h!tb]
  \center
  \includegraphics[width=8cm]{figs/ex_b_angularmomentum_velocityverlet_1.pdf}
  \includegraphics[width=8cm]{figs/ex_b_angularmomentum_eulerforward_1.pdf}
\end{figure}

\begin{figure}[h!tb]
  \center
  \subfloat[][]{\includegraphics[width=14cm]{figs/exe_earth_jupiter_10.pdf}}\\
  \subfloat[][]{\includegraphics[width=14cm]{figs/exe_earth_jupiter_1000.pdf}}
\end{figure}

\begin{figure}[h!tb]
  \center
  \includegraphics{figs/all_planets3d.pdf}
\end{figure}




\section{Conclusions}
  In trying to generalize my code through the use of object orientation, i had to strike a balance between generality and complexity

\begin{thebibliography}{99}
\bibitem{lecture_notes} M.~Hjorth-Jensen, Computational Physics - Lecture Notes 2015, (2015).
\bibitem{problem_set} M.~Hjorth-Jensen, Building a model for the solar system using ordinary differ-
ential equations - Project 3 (2018)
\end{thebibliography}

\appendix
\section{Fetching data from Horizons using horizons.py}
  In order to streamline the process of fetching data from the Horizons system i created a small script, \lstinline{horizons.py} that utilizes the Astroquery python package and returns only the select data that i am interested in. The function, \lstinline{fetch_data} takes input as a dictionary, rather than a list of id numbers. Whilst this may not be as extensible or practical, it makes the code easier to read and understand, by making the instant connection between the planet name and id. For the purposes of this project, i find that much more valuable since i am only dealing with a limited amount of planets anyway.


\end{document}  