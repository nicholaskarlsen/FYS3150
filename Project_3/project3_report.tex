\documentclass[10pt,showpacs,preprintnumbers,footinbib,amsmath,amssymb,aps,prl,twocolumn,groupedaddress,superscriptaddress,showkeys]{revtex4-1}
\usepackage{graphicx}
\usepackage{dcolumn}
\usepackage{bm}
\usepackage[colorlinks=true,urlcolor=blue,citecolor=blue]{hyperref}
\usepackage{color}
\usepackage{listings}

\lstset{ %
  basicstyle=\footnotesize,        % the size of the fonts that are used for the code
  breakatwhitespace=false,         % sets if automatic breaks should only happen at whitespace
  breaklines=true,                 % sets automatic line breaking
  captionpos=t,                    % sets the caption-position to bottom
  deletekeywords={...},            % if you want to delete keywords from the given language
  escapeinside={\%*}{*)},          % if you want to add LaTeX within your code
  extendedchars=true,              % lets you use non-ASCII characters; for 8-bits encodings only, does not work with UTF-8
  frame=single,                    % adds a frame around the code
  keepspaces=true,                 % keeps spaces in text, useful for keeping indentation of code (possibly needs columns=flexible)
 % language=Python,                 % the language of the code
  morekeywords={*,...},           % if you want to add more keywords to the set
  numbers=left,                    % where to put the line-numbers; possible values are (none, left, right)
  numbersep=5pt,                   % how far the line-numbers are from the code
  showspaces=false,                % show spaces everywhere adding particular underscores; it overrides 'showstringspaces'
  showstringspaces=false,          % underline spaces within strings only
  showtabs=false,                  % show tabs within strings adding particular underscores
  stepnumber=1,                    % the step between two line-numbers. If it's 1, each line will be numbered
  tabsize=2,                       % sets default tabsize to 2 spaces
}


\begin{document}
\title{FYS3150 Computational Physics - Project 3}
\author{Nicholas Karlsen}

\begin{abstract}
This is an abstract
\end{abstract}

\maketitle

\section{Introduction}
  
  Lastly, the source code for any code discussed in this report can be found on my
  Github at: \url{https://github.com/nicholaskarlsen/FYS3150}

\section{Theory, Algorithms and Methods}
\begin{lstlisting}[mathescape=true, language=python, title=Euler-Cromer Algorithm]
for $i = 0, \dots, N$
    $\mathbf v_{i + 1} = \mathbf v_{i} + \mathbf a_{i}\Delta t$
    $\mathbf r_{i + 1} = \mathbf r_{i} + \mathbf v_{i + 1}$
\end{lstlisting}

\begin{lstlisting}[mathescape=true, language=python, title=Velocity-Verlet Algorithm]
for $i = 0, \dots, N$
  $\mathbf r_{i+1} = \mathbf r_i + \mathbf v_i \Delta t + \frac{1}{2m}\mathbf F(t_i)\Delta t^2$
  $\mathbf v_{i+1} = \mathbf v_i + \frac{1}{2m}\left( \mathbf F(t_i) + \mathbf F(t_{i+1}) \right)\Delta t$
\end{lstlisting}

\section{Results and Discussions}

\section{Conclusions}


\begin{thebibliography}{99}
\bibitem{lecture_notes} M.~Hjorth-Jensen, Computational Physics - Lecture Notes 2015, (2015).
\end{thebibliography}

\end{document}  