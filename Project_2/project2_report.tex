\documentclass[10pt,showpacs,preprintnumbers,footinbib,amsmath,amssymb,aps,prl,twocolumn,groupedaddress,superscriptaddress,showkeys]{revtex4-1}
\usepackage{graphicx}
\usepackage{dcolumn}
\usepackage{bm}
\usepackage[colorlinks=true,urlcolor=blue,citecolor=blue]{hyperref}
\usepackage{color}
\usepackage{listings}

\lstset{ %
  basicstyle=\footnotesize,        % the size of the fonts that are used for the code
  breakatwhitespace=false,         % sets if automatic breaks should only happen at whitespace
  breaklines=true,                 % sets automatic line breaking
  captionpos=t,                    % sets the caption-position to bottom
  deletekeywords={...},            % if you want to delete keywords from the given language
  escapeinside={\%*}{*)},          % if you want to add LaTeX within your code
  extendedchars=true,              % lets you use non-ASCII characters; for 8-bits encodings only, does not work with UTF-8
  frame=single,                    % adds a frame around the code
  keepspaces=true,                 % keeps spaces in text, useful for keeping indentation of code (possibly needs columns=flexible)
 % language=Python,                 % the language of the code
  morekeywords={*,...},           % if you want to add more keywords to the set
  numbers=left,                    % where to put the line-numbers; possible values are (none, left, right)
  numbersep=5pt,                   % how far the line-numbers are from the code
  showspaces=false,                % show spaces everywhere adding particular underscores; it overrides 'showstringspaces'
  showstringspaces=false,          % underline spaces within strings only
  showtabs=false,                  % show tabs within strings adding particular underscores
  stepnumber=1,                    % the step between two line-numbers. If it's 1, each line will be numbered
  tabsize=2,                       % sets default tabsize to 2 spaces
}


\begin{document}
\title{FYS3150 Computational Physics - Project 2}
\author{Nicholas Karlsen}

\begin{abstract}
  This is an abstract
\end{abstract}

\maketitle

\section{Introduction}
  \subsection{Preservation of scalar product \& orthogonality in unitary transformations}
    Consider an orthonormal set of basis vectors $\mathbf v_i$ such that $\mathbf v_j^T \mathbf v_i = \delta_{ij}$. Let unitary matrix $U$ where $U^T U= I_N$, where $I_N$ denotes the $N\times N$ identity matrix, operate on $\mathbf v_i$ to get $\mathbf w_i$
    \begin{equation}
      \mathbf w_i = U \mathbf v_i
    \end{equation}
    Then
    \begin{equation}
      \mathbf w_j^T\mathbf w_i = (U\mathbf v_j)^TU\mathbf v_i = \mathbf v_j^T U^T U \mathbf v_i
      = \mathbf v_j^T \mathbf v_i = \delta_{ij}
    \end{equation}
    In the unitary transformation of $\mathbf v_i$ both the scalar product and orthogonality has been preserved.
\section{Theory, Algorithms and Methods}
\section{Results and Discussions}
\section{Conclusions}

 

\begin{thebibliography}{99}
\bibitem{lecture_notes} M.~Hjorth-Jensen, Computational Physics - Lecture Notes 2015, (2015).
\end{thebibliography}


\end{document}